


\section*{entire facts
	pole at \infty iff polynomial (this further implies infinitely many times through PIcard theorem)
	after proving constant/poly, check if the bound implies values at special point (e.g. origin), this may further constain the form
	dense image
	essential singularity prevent injective, and the only injective polynomial is linear 
	removable singularity also prevent injective through Liouville
	
\section*{holomorphic facts
	e^f, (1+i)f
	inversion trick (1/f, 1/(f(z)-w))
	divide zero trick
	multiply pole trick
	subtract zero/pole trick
	no zeros, 
		must equal to e^h
		with equality bdd -> constant
	1/z transform 
		preserve injectivity
		zero, pole location change, property does not change
	1/f
		zero, pole locaiton does not change, property change
	
	
\section*{harmonic facts
	constant shift
\section*{derivative tricks
	integral and residue theorem
	cauch estimate
	special FTC integral z\int_0^1f'(tz)
	weierstrass second part
	order equality
	derivative value of \phi_a/schwarz
	fe^{-F}, where F=f'/f+...
\section*{integral tricks
	FTC
	MVP (can used for bdd, applicable to both harmonic/holomorphic)
\section*{automorphism trick
	exp strip log slit
	5 pt map
	sin/cos/tan
\section*{bounded
	removable singualrity (if bdd near singu as well)
	/Jensen's formula, especially about zero distribution/division trick, cauchy estimate
	blaschke product
	1/z and z^n multiply/division trick does not change bound on unit circle
	pointwise bound (of sin/cos/exp)
	schwarz/biholomorphisms
	order (derivative/integral bdd)
	
\section*{reverse bounded
	rouche's
	cauchy estimate: If M(r) \geq ..., then there exists |f| \geq ..
	m-test
	mmp (with inverse trick)
	entire lower bdd by poly must be poly (as \to \infty pole)
	
\section*{meromorphic facts
	related to rational, multiply a polynomial containing all poles then entire function, (and if pole at \infty, then polynomial)
	discrete pole, hence for any pole, there is a holomorphic neighborhood
	
\section*{polynomial
	finitely many roots, pole at infinity
	
\section*{convergence/sequence trick
	M-test/root test/geomtric R<1/power >1
	hurwitz/\epsilon-\delta proof
	weierstrass
	Montel
	integral, residue theorem

\section*{order trick (check if universal)
	if order \lambda, then (equivalently)
		1. for any \epsilon, there exists R big enough such that |f| \leq e^{R^{\lambda+\epsilon}}, plug this in and later argue \epsilon arbitary
		2. for any \epsilon, there exists R big enough such that logM(R) \leq R^{\lambda+\epsilon}, plug this in and later argue \epsilon arbitary
	
	sum/product \leq max order (need proof)
	primitive/derivative same order
	identity theorem of some sort can be achieved thorugh hadamard (rank order conflict)
	weierstrass form constraint
	
	
	
	
\section*{====various theorem
\section*{cauchy integral formula/MVP/cauchy estimate/Jensen --> holomorphic on both interior and trajectory (boundary)
	!!! if not, prove/make those points are removable singularity first
	!!! ML inrquality reached iff integrand constant
	
\section*{residue theorem --> holomorphic on curve, finitely many pole in interior
\section*{simply connected -> open/connected
\section*{M-test uniformly and absolutely (within the set we apply, but we usually need a compact set to find bound....)





\section*{rst/casorati/picard ---> punctured neighborhood
ftc --> integrand holomorphic in an neighborhood of the curve (so that its integrand is well defined)


\section*{convergence
converge uniformly in the entire domain (limsup)--->
converge locally uniformly (i.e. uniformly in each open neighborhood) 
= converge on compact (i.e. on each compact, usually can be bounded by radius R)
---> integral within the set converges (through ML argument)
	!!! after proving, need to justify why interior to interior (OMT/MMP) and why not a constant
	!!! all the harmo/hol/subharmo uses l.u.
conversely integral not converge
---> not l.u/u_compact/uniform approximation
	
\section*{MMP/OMT --> always usable, if contradict, the function will be constant (may not be the end of argument, may be given an extra value to show that such constant cannot even exist)




\section*{rouche's theorem ---> 
1. holomorphic on both interior and boundary
2. strict inequality on boundary 
3. curve no self-intersection

\section*{hurwitz theorem ---> 	
1. holomorphic
2. locally uniformly convergence
3. compact subset (completely inside domain)
4. target no zero on the boundary of that compact subset

\section*{weierstrass theorem ---> holomorphic + locally uniformly convergence
	!!! does not determine the image of target function, need to use OMT/MMP to ensure interior to interior

\section*{identity theorem ---> holomorphic in an open set that contain the equality sequence INCLUDING its accumulation pt

\section*{infinite product converges locally uniformly (and absolutely)--> sum converges locally uniformly (and absolutely)
	!!! M-test exactly provides that

\section*{Montel ---> locally converges iff locally bdd
	need to argue neighborhood of 0/\infty



\section*{schwarz lemma --> 3 stage
condition stage 1: D->D holomorphic, f(0)=0
condition stage 2: automorphism 
condition stage 3: one point value other than origin known

\section*{schwarz reflection
1. holomorphic in sided interior
2. continuous, real value on real line (or modulus 1 on circle)

	real line: inner conjugate, outer conjugate
	circle: inner inverse conjugate, outer inverse conjugate
	when inductive reflect, fix one side and keep merging

\section*{RMT --> non-empty, simply connected set

\section*{Runge
1. compact
2. holomorphic on that compact (i.e. an open set containing that compact)
3. AT LEAST one point from each bounded hole

(then uniform appx on that compact)

1. open
2. holomorphic on that open
3. AT LEAST one point from each bounded hole

(then locally uniform appx on that open)

	1. does not care disjoint or not, still one approximation for all
	2. if two funcitons are the same on a set, then approximation for one is an approimation for another
	
	
move special bbd trick???	
	
\section*{simply connected --> open/connected



\section*{weierstrass/ML --> always usable (entire only)
\section*{hadamard ---> non-zero (entire only), if contradiciton, then constant (may not be the end of proof)
\section*{little picard ---> basic version: entire
---> special version: meromorphic
\section*{great picard ---> basic version: entire with singularity
---> special version: entire non-polynomial
---> both assume non-constant, hence if contradiction, then constant

\section*{blaschke product --> defined on both interior and boundary

