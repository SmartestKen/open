\section*{holomorphic tricks}
1. polynomial tricks
	any bound is bounded, and we can always use cauchy's estimate.
	if the bound itself goes to 0, then the entire function is identically 0

	lower bound
		>= |z|^N ----> polynomial
		

	
	
	
	sequence + zeros ---> hurwitz
	

2. rational tricks

3. general holomorphic tricks

	identity theorem
	
	proving k = k' is the key to k = ce^z


16,15,14,13,12,

\section*{sequences trick}
	Montel
	
	Hurwitz, when dealing with 0
	
	\epsilon-\delta argument, uniformly continuity, when dealing with inverse
	
	Weierstrass theorem

\section*{proving bdd tricks}
	pointwise bdd for e/sin/cos

\section*{using bdd tricks}
	to reverse the bound, try 1/f or -f. 
		if we do 1/f, we can consider multiply a polynomial that cancel all of f's root. BUt to do that, we need to ensure f does not have zero sequence that goes to infinity (i.e. number of zeros in a bounded set must be finite, otherwise use identity theorem)



	


\section*{mapping, classify, existence}
	open mapping theorem 
	preimage argument




\section*{harmonic tricks}
1. locally (i.e. an open neighborhood) construct and use holomorphic related results

2. MVP and its disk integral version
	2.1 green theorem: turn line integral into disk integral by adding derivative (and vice versa)

3. u_xx + u_yy = 0

4. maximum modulus principle

5. cauchy estimate 
	we must have
	1. a bound 
	2. a radius such that the corresponding disk is inside the applicable area of that bound (hence it does not have to be the maximum radius)

lowest term of series decided by lowest terms from component. 
Given a series, we can always suppose the lowest non-zero term is blabla...

!!! (key trick) nonconstant entire function dense image

nowhere zero except for finitely many puncture -> nowhere zero
(1-i)f
e^f

FA20 P4

FA17 P8 does that mean we have to deveop the machinery
FA17 p6
220A 11 --> bdd

omit values -> oomit special values
rational manipulation does not change order
entire, non-zero , then e^h for some other entire h
hadamard contradiction ---> the function we apply hamadard is zero. (i.e. DO NOT JUST CONCLUDE AT HADAMARD CONTRADICTION)

!!! cauchy estimate can ONLY be applied when the function is holomorphic everywhere in the disk (and circle). If there is a pole/unknown point, deal with it first, DO NOT JUST CLAIM that it is a polynomial.

rational verion of cauchy??? fill in things there?



convert an meromorphic to entire
	1. use to omitted value to prevent blow
	2. assign 0 at the original poles

hence picard tells us for entire to be constant, we need 2 omission. for meromorphic to be cosntant, we need 3 omission

value at the set*measure at the set = integral
1/g automatically omit 0 for any entire g.
exp automatically omit 0

whenever possibility of rational or negative power, convert it into a positive power and 
	1. removable singularity
	2. cauchy estimate


1/f(z) zero-> pole, pole->removable singulairty (0), location does not change
f(1/z) zero/pole property does not change, but location swap

(key property) entire f(z) with removable singularity/pole at \infty is a polynomial
hence entire f(1/z) with removable singularity/pole at 0 is a polynomial
hence entire, nonpolynomial f(1/z) must have essential singularity at 0
entire, nonpolynomial f(z) must have essential singularity at \infty

Hence great picard, any entire, non-polynomial (e.g. exp^{anything other than constant}) must take any value, with one exception, infinitely many times.

hence entire + omit one value + attain another value finitely many times -> that entire funciton must be a polynomial
	!!! hence if that function is non-constant of the form exp(...), then contradiction.
entire +omit two value -> that entire function is a constant
meromorphic +omit 3 value -> constant

g no fix point, then g(z)-z omit 0, g(g(z)) - g(z) omit 0 and so on.
if f has fix point, then f \circ f has fix point
(i.e. any $\neq$ can be converted into an omission)

if we want to deal with omission of f and the equation we have only have (f+c)*..=,..., then we might as well call h = f+c and analyze h instead. Now if we get two omission from h, it will be the same as we get two omission from f.
in fact, any property we obtain from (f-a)/b transfer exactly back to f

!!! f' \circ f (or any composition) never attain what f' cannot attain











if omit and finite exception appear at the same time, then try to split tehm up????

!!! ce^{non-constant} = polynomial ,then c = 0

that one exception can be taken on finitely many times, e.g. (1-z)e^{1/z} can attain 0 once.


1. prove eachpartial sum subharmonic
2. prove convergence
then target is subharmonic (as limit commute with mvp)

!!! DO NOT DIRECTLY SWAP the infinite sum with the integral, use the argument (each finite partial sum subharmonic + convergence) above

in general, when proving infinite sum harmo/holo, always argue on finite partial sum, then push limit (through convergence theorem, eg.. weierstrass, the one for harmo and subharmo)

type of bound:
	bound involving integral
	bound invovling derivative
	bound involving negative power
	bound involving log
	bound involving polynomial
	constant bound
	reverse bound (i.e. \geq rather than \leq)





!! one way is to use zf(z), as z->0, limzf(z) = 0, hence fill in 0 at there.
the other (completely equivalent) way is to directly use statement of removable singularity theorem, limzf(z) = 0 MEANS f(z) (rather tahn zf(z)) has a removable singlarity. Cauhcy still useable as negative power does not infleucne as R -> \infty

!!! E(...), whatever inside the bracket multiply its probability
!!! E(..|..), whatever on the left side of bracket, multiply its conditional probability.




hadamard helps determine the form on the exponent. without that it is hard to argue h/log(R) \infty for some random non-constant h. (i.e. we can only derive order when we completely confirm the exact expression of our entire function)


e^{polynoial} --> order= degree of poly
polynomial --> order= 0 
e^e^{poly}} ---> order = \infty

submvp: non-integral side \leq integral side (hence given any condition of the form, start with non-integral side, reverse the equation if neeeded)
mvp, integrating counterclockwise/clockwise does not matter

learn the odd bound residue details??
bounded everywhere except for those singularities -> those singularities must be removable singularity, fill them in and we get a constant function
!!! OMT/MMP can only be used to argue that f can either map to interior or a cosntant (on the boundary). Yet if the question require us to prove that f must map to the interior, OMT/MMP alone does not help.

finish first pass, then second pass make hierarchy (summary) node

\section*{to reverse the bound, try 1/f} 
if we do 1/f, we can consider multiply a polynomial that cancel all of f's root. BUt to do that, we need to ensure f does not have zero sequence that goes to infinity (i.e. number of zeros in a bounded set must be finite, otherwise use identity theorem)
















\section*{harmonic}
1. MVP (basic and disk version).  (220C HW1.3)
	!!! commute with taking derivative and limit
	!! Green's theorem connect disk integral of partial derivative with line integral of itself.

	1.1 cauchy estimate/Liouville (harmonic version)

2. also given bdd from above (220C HW1.4)
	use e^f (which automatically provide the bound from below)
	!!! if given bdd from below, consider -u


3. \Delta u = u_{xx} + u_{yy} = 0		(220C HW2.1, HW2.2)
	
	3.1 punctured disk at center, h(r) = alog(r)+b = MVP integral for some constant a and b.
	
	3.2 removable singularity fillable (proven using 3.1 and CONTINUITY)

4. e.g. (and the only e.g.) log|z| (220C HW2.3)
	e.g. i(2\pi)\theta   (220C HW2.4)


5. reflection principle (220C HW2.5)
	proven using 3, note that du(x,-y)/dy = -u(x,-y) by chain rule




\section*{harmo-holo connection}
1. transfer to holomorphic (prove property and transfer back) (220C HW1.1, HW1.2)		
	1. locally if the domain is open
	2. globally if domain is simply connected
	
	!!! we can also explicitly construct holo to SHOW harmo

	1.1 composition rule
		!!! can be used to show that log|h(z)| is harmonic
	1.2 Opem mapping theorem (harmonic version)



2. admit conjugate (and therefore admit a corresponding holomorphic) (220C HW1.5)
if and only if every nowhere-zero holomorphic can be expressed a e^h 
if and only if simply connected



3. if two corresponding holomorphic, then up to an imaginary constant. (220A HW2.3)
	!!! conversely, if we obtain a holomorphic from 3, we can always adjust by an imaginary constant and it is still a corresponding holomorphic








\section*{show constant}
1. show modulus is a constant, then use OMT (especially when domain is the entire complex plane) or MMP Iwhen the domain is bounded)

2. find a bound, and use cauchy estimate/Liouville. We need to note about neighborhood of infinity is covered if we construct that bound ourselves.

\section*{use/prove simply connected}
1. (if and only if) given any non-zero holomorphic h, h can be written as e^g
	!!! to construct consider showing |he^-g| = 1




\section*{cauchy estimate}
1. To use it
	1. find a bound
	2. find a disk covered by the bound's applicable area (use MMP if you do not choose the maximal disk)

\section*{bound}
1. polynomial/log(polynomial) bound
	just use cauchy estimate (applicable to both harmonic and holomorphic)


\section*{exist function}
1. Weierstrass
	log derivative can become a solution to ML
2. ML
 	!!! note that this gives more control on construction than Weierstrass. Use ML whnever possible (e.g. f,g entire, f/g meromorphic)
3. combinging weierstrass with zero of order m and ML (with m terms)
	1. the product is entire
	2. we can decide the first m terms (that is, up z^{m-1}) of its taylor series





2. logarithmic derivative of Weierstrass (220B.3.3)


\section*{show/use normal}
1. Montel theorem (to jump from/to locally uniformly bdd)


\section*{prove construction equality}
1. show that both side have the same zeros
2. log derivative (by adding e^h to one side) and use special value to show h=0
or
2. use special property along the boundary (e.g. blaschke), and MMP (f/g, g/f)

\section*{use conformal mapping/biholomorphism}
basic property
1. EACH generalized circle (that is LINE or CIRCLE) in src boundary mapped to one of generalized circle in target boundary, 1 extra point to decide mapped to inside or outside
2. preserve the intersection angle of generalized circles



!!! Remember e_strip_log_slit

To prove mapping works:		(220C HW2.3)
	1. for boundary to boundary, use conformal mapping (pick 3 points and show generalized circle -> generalized circle)
	2. for interior to interior, suppose interior to boundary and then OMT. Or in case we cannot determine which side of target boundary (use extra pt or MMP)


\section*{int/deriv tricks}
(220C HW2.1, HW2.2)

dxdy = rdrd\theta
\Delta u 
= u_{xx} + u_{yy} 
= u_{rr} + u_r/r + u_{\theta\theta}/r^2
= ((r(d/dr))^2 + d^2/d\theta^2)u/r^2

!!! close loop + fundamental theorem (primitive exists) -> 0 result. (Now note that the primitive of u_{\theta\theta} is EXACTLY u_{\theta} when inetgrating with respect to d\theta)


!!! derivative non-negative --> self non-decreasing
derivative 0 -> self constant

!!! d/dz = d/dx - id/dy
d/d\bar{z} = d/dx + id/dy ---> which is why df/d\bar{z} = 0 for any holomorphic f
d/d\bar{z}d/dz|f|^2 = |f'|^2 (by product rule)


!!! u_x = v_y (left, left, right, right)

\section*{int technique}

\section*{subharmonic}
1.(if and only if) \Delta u \geq 0
	
2. (if and only if) subMVP (with range)






\section*{220B HW1, HW2, HW3}
\section*{220A HW1, HW2, }
\section*{220C HW1, }


!!! the only time write is the NEW understanding flush

\section*{schwarz reflection}
!! harmonic version can directly proven, no need to jump back to holo version (unlike OMT)

!!! DO NOT directly try to prove MVP, as it has to be any radius and we are to prove harmonic not to use harmonic.


!! converge locally uniformly -> converge uniformly on compact -> integral (which is contained inside that compact set) converges -> therefore inequality passes to there.


\section*{OMT/MMP expand}
interior point cannot be mapped to boundary
modulus bound for boundary is bound for interior

src open cannot map to boudnary (not open)
open domain cannot be mapped to closed domain (e.g. circle of fixed modulus)


!! bounded from below
	1. inverse both side to get \leq
	2. use the fact that entire \to \infty (i.e. with a pole at \infty) must be polynomial
